
%\documentclass[12pt, conference, compsocconf ]{IEEEtran}
\documentclass[12pt]{article}
\usepackage{hyperref}
\usepackage{amsmath}
\usepackage{listings}
\usepackage{graphicx}
\usepackage{verbatim}


\title{INF3490 \\ Supervised Learning with Multilayer Perceptron}
\author{
Mandatory Assignment 2\\
Siddhartha Pandey\\
\href{mailto:siddharp@ifi.uio.no}{siddharp@ifi.uio.no}
}

\begin{document}
\maketitle

%\begin{abstract}

%\end{abstract}


\section*{Introduction}
This assignment uses supervised learning with a Multi-layer perceptron (MLP) to classify electromyographic signals to its corresponding hand movement. 

%\newpage

\section*{Program}

The program was coded using python. We use \textit{movements.py} to read the data from the given data files. Then from  \textit{movements.py} the data is trained and tested using \textit{mlp.py}. 

We use 6, 8, 12 and 18 nodes in the hidden layer to test our program and the result obtained can be seen in the next section.

\section*{Result} 

The following data shown are the results from running the MLP network on the given data sets.
The result shown are the confusion table obtained from running the program on the given data sets and the correct percentage.

\verbatiminput{test_data.txt}


\subsection*{Hidden Nodes : 6}  

\subsubsection*{Iteration:100}
\paragraph*{Confusion table :}

\[
  \begin{matrix}
   & class1 & class2 & class3 & class4 & class5 & class6 & class7 & class8 \\
class1 &    10 & 0 & 0 & 0 & 0 & 0 & 0 & 0 \\
class2 &    0 & 9 & 0 & 0 & 0 & 2 & 0 & 0 \\
class3 &    0 & 0 & 8 & 1 & 0 & 0 & 0 & 0 \\
class4 &    0 & 0 & 0 & 23 & 0 & 0 & 2 & 0 \\
class5 &    1 & 0 & 0 & 3 & 16 & 1 & 0 & 0 \\
class6 &    0 & 0 & 0 & 0 & 0 & 8 & 0 & 0 \\
class7 &    0 & 0 & 0 & 0 & 0 & 1 & 12 & 0 \\
class8 &    0 & 0 & 0 & 1 & 0 & 0 & 0 & 13 \\
  \end{matrix}
\]

\paragraph*{Percentage Correct: 89.91} 

\subsection*{Hidden Nodes : 8}  


\subsubsection*{Iteration:100}

\paragraph*{Confusion table :}

\[
  \begin{matrix}
   & class1 & class2 & class3 & class4 & class5 & class6 & class7 & class8 \\
 class1 &   10 & 0 & 0 & 0 & 0 & 0 & 0 & 0 \\
 class2 &   0 & 9 & 0 & 0 & 0 & 2 & 0 & 0 \\
 class3 &   0 & 0 & 8 & 0 & 0 & 0 & 0 & 0 \\
 class4 &   0 & 0 & 0 & 24 & 0 & 0 & 0 & 0 \\
 class5 &   1 & 0 & 0 & 2 & 16 & 0 & 0 & 0 \\
 class6 &   0 & 0 & 0 & 0 & 0 & 9 & 0 & 0 \\
 class7 &   0 & 0 & 0 & 0 & 0 & 1 & 14 & 0 \\
 class8 &   0 & 0 & 0 & 2 & 0 & 0 & 0 & 13 \\
  \end{matrix}
\]

\paragraph*{Percentage Correct: 92.79}  

\subsection*{Hidden Nodes : 12}  

\subsubsection*{Iteration:100}
\paragraph*{Confusion table :}

\[
  \begin{matrix}
   & class1 & class2 & class3 & class4 & class5 & class6 & class7 & class8 \\
class1 &    10 & 0 & 0 & 0 & 0 & 0 & 0 & 0 \\
class2 &    0 & 9 & 0 & 0 & 0 & 2 & 0 & 0 \\
class3 &    0 & 0 & 8 & 1 & 0 & 0 & 0 & 0 \\
class4 &    0 & 0 & 0 & 24 & 0 & 0 & 0 & 0 \\
class5 &     1 & 0 & 0 & 3 & 16 & 1 & 0 & 0 \\
class6 &    0 & 0 & 0 & 0 & 0 & 9 & 0 & 0 \\
class7 &    0 & 0 & 0 & 0 & 0 & 0 & 14 & 0 \\
class8 &    0 & 0 & 0 & 0 & 0 & 0 & 0 & 13 \\
  \end{matrix}
\]
\paragraph*{Percentage Correct: 92.79} 
\subsection*{Hidden Nodes : 18}  

\subsubsection*{Iteration:100}
\paragraph*{Confusion table :}

\[
  \begin{matrix}
  & class1 & class2 & class3 & class4 & class5 & class6 & class7 & class8 \\
class1 &    10 & 0 & 0 & 0 & 0 & 0 & 0 & 0 \\
class2 &    0 & 9 & 0 & 0 & 0 & 2 & 0 & 0 \\
class3 &    0 & 0 & 8 & 1 & 0 & 0 & 0 & 0 \\
class4 &    0 & 0 & 0 & 24 & 0 & 0 & 0 & 0 \\
class5 &    1 & 0 & 0 & 3 & 16 & 1 & 0 & 0 \\
class6 &    0 & 0 & 0 & 0 & 0 & 7 & 0 & 0 \\
class7 &    0 & 0 & 0 & 0 & 0 & 2 & 14 & 0 \\
class8 &    0 & 0 & 0 & 2 & 0 & 0 & 0 & 13 \\
  \end{matrix}
\]
\paragraph*{Percentage Correct: 90.99} 

\section*{}

Hidden Layer with the greater no. of nodes that is 12 and 18 have higher correct percentage than with less no. of nodes.
By looking at the confusion tables given above we can say that class 4 and class 5 were likely to mistaken for each other.



\end{document}










































